\begin{frame}[t]{Resultados}
    \transboxout[duration=0.5]
    \framesubtitle{Inovações} 
O dispositivo apresentou duas propriedades:
\begin{itemize}
    \item Bits definindo propriedades mecânicas
    \item O estado dos bits poder ser controlados remotamente por campos magnéticos 
\end{itemize}
\only<2->{
    \begin{alertblock}{}
        Nunca antes um dispositivo demonstrou ter estas duas propriedades simultâneamente.
    \end{alertblock}
}
 %*----------- notes
    \note[item]{Notes can help you to remember important information. Turn on the notes option.}
\end{frame}

\begin{frame}[t]{Resultados}
    \transboxout[duration=0.5]
    \framesubtitle{Limitações} 
    \begin{itemize}
        \item Bits mecânicos possuem uma geometria complexa\\
        \only<2->{Por conta disso, há a necessidade de ser feito de materiais moles.}
            \item O dispositivo possui apenas 36 bits\\
            \only<2->{Não é eficiente para aplicações práticas, sendo necessário uma miniaturização e uma transformação em um modelo 3D.}
    \end{itemize}
    \note[item]{Materiais moles que apresentam ferromagnetismo,
        o tipo 'convencional' de magnetismo encontrado em
        ímãs de ferro.}
\end{frame}

\begin{frame}
    %\transdissolve[duration=0.5]
    %\hspace*{-1cm}
    \begin{columns}
        %\column{.01\textwidth}
        \column{0.4\textwidth}
            ~\hfill
            \vbox{}\vskip-1.4ex%
            \begin{beamercolorbox}[sep=8em, colsep*=18pt, center, wd=\textwidth, ht=\paperheight]{title page header}%
                \begin{center}
                    \textbf{\huge{Conlusão}}\par
                
                \end{center}
            \end{beamercolorbox}%
        \hfill\hfill
        \column{.05\textwidth} 
        \column{.6\textwidth}
        O dispositivo demonstrou a possibilidade de:
        \begin{itemize}
            \item Codificar informações
            \item Armazená-las permanentemente
            \item Determinar propriedades mecânicas de um corpo
        
        \end{itemize}
    \end{columns}
  
 %*----------- notes__
    \note[item]{Notes can help you to remember important information. Turn on the notes option.}
\end{frame}

\begin{frame}
    %\transdissolve[duration=0.5]
    %\hspace*{-1cm}
    \begin{columns}
        %\column{.01\textwidth}
        \column{0.4\textwidth}
            ~\hfill
            \vbox{}\vskip-1.4ex%
            \begin{beamercolorbox}[sep=8em, colsep*=18pt, center, wd=\textwidth, ht=\paperheight]{title page header}%
                \begin{center}
                    \textbf{\huge{Conlusão}}\par
                
                \end{center}
            \end{beamercolorbox}%
        \hfill\hfill
        \column{.05\textwidth} 
        \column{.6\textwidth}
        No entanto, ainda há limitações, por conta:
        \begin{itemize}
            \item Complexidade da estrutura
            \item Dificuldade na produção em larga escala
        \end{itemize}
    \end{columns}
  
 %*----------- notes__
    \note[item]{Notes can help you to remember important information. Turn on the notes option.}
\end{frame}

\begin{frame}
    %\transdissolve[duration=0.5]
    %\hspace*{-1cm}
    \begin{columns}
        %\column{.01\textwidth}
        \column{0.4\textwidth}
            ~\hfill
            \vbox{}\vskip-1.4ex%
            \begin{beamercolorbox}[sep=8em, colsep*=18pt, center, wd=\textwidth, ht=\paperheight]{title page header}%
                \begin{center}
                    \textbf{\huge{Conlusão}}\par
                
                \end{center}
            \end{beamercolorbox}%
        \hfill\hfill
        \column{.05\textwidth} 
        \column{.6\textwidth}
        Demonstrou ser uma solução para aplicações que necessitam:
        \begin{itemize}
            \item Controlar rigidez de materiais em diferentes partes do corpo
            \item Controle da densidade de energia armazenada por um material        
        \end{itemize}
    \end{columns}
  
 %*----------- notes__
    \note[item]{Notes can help you to remember important information. Turn on the notes option.}
\end{frame}

\begin{frame}[t]{Conlusão}
    \transboxout[duration=0.5]
    \framesubtitle{Trabalhos futuros} 
    \note[item]{Como este trabalho focou em trabalhar as duas propriedades básicas dos mateiriais: rigidez e densidade de energia.}
    \begin{itemize}
        \item Controle de propagação de ondas
        \item Dissipação de energia
        \item Formas complexas de deformação, como mudança de textura
    \end{itemize}
\end{frame}
   