\section{Resultados}

\subsection{Inovações}
Este dispositivo traz uma grande contribuição para a comunidade, pois foi o primeiro metamaterial que apresentou simultâneamente:

\begin{itemize}
    \item Bits definindo propriedades mecânicas
    \item O estado dos bits podem ser controlados remotamente
\end{itemize}

\subsection{Limitações}
Apesar de apresentar diversas inovações, o dispositivo ainda possui algumas limitações que ainda restrigem de elaborar aplicações práticas. Pois:
\begin{itemize}
    \item Geometria complexa do material
    \item Densidade de armazenamento\footnote[1]{São necessários 10mm² para obter 1 bit, obtendo 36 bits com 1080mm².}
\end{itemize}

Por ter uma geometria complexa, se tornar dificil o processo de miniaturizar estas estruturas. O que impacta no segundo ítem, que é o fato de ter uma densidade de armazenamento extremamente baixa, pois para ter um bit é necessário ter uma área de 10mm².

