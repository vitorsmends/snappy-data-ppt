\section{Funcionamento}

\subsection{Unidade Básica de Armazenamento}
\begin{figure}[H]
    \centering
    \includegraphics[scale = 0.5]{source/pictures/shell.png}
    \caption{Unidade básica de armazenamento mecânico\cite{coulais2021snappy}.}
    \label{fig:uba}
\end{figure}

A estrutura da unidade básica de armazenamento mecânica possui esta estrutura, que é projeta para ter uma instabilidade de encaixe. Ela conta com dois \textit{sttopers}, que funcionam como braços para limitar a movimentação da estrutura interna e uma tampa magnética. Através dela, é possível orientar remotamente o encaixe da estrutura, por meio de um campo magnético.


\begin{figure}[H]
    \centering
    \includegraphics[scale = 0.5]{source/pictures/shell.png}
    \caption{Estados da UBA\cite{coulais2021snappy}.}
    \label{fig:uba-states}
\end{figure}

Nesta imagem(\ref{fig:uba-states}) é possível ver como a estrutura se altera conforme uma mudança na orientação magnética da tampa. A partir disso, é possível definir se o material é mais facilmente comprimido ou o contrário.

\subsection{Processo de gravação}

Semelhante ao disco rígido, onde o cabeçote eletromagnético orienta a polaridade da unidade básica de memória. Neste caso, um cabeçote eletromagnético vai ser responsável por orientar o estado da estrutura, como foi possível ver anteriormente.

\begin{center}
\textcolor{red}{\textbf{[VIDEO DE GRAVAÇÃO EM UBA]}}
\end{center}

Neste vídeo, é possível analisar a orientação do encaixe em uma unidade, a partir criação de um campo eletromagnético.

\begin{center}
\textcolor{red}{\textbf{[VIDEO DE GRAVAÇÃO EM ARRAY]}}
\end{center}


